\section*{Forord}
P� kandidatdelen af �konomistudiet p� K�benhavns Universitet  faldt min 
interesse p� boligmarked, bl.a. grundet den �get fokus p� boligmarkdet i �konomiske 
analyser efter finanskrisen. Gennem mit studiejob i Bolig�konomisk Videncenter var 
jeg bl.a. st�dt p� analyser at boligmarked, der estimerede markedet ved hj�lp af hedoniske 
prisfasts�ttelsesmetoder. Efter at have studerede den hedoniske metode n�rmer, 
samt �get fokus i mediebillede p� metrobyggeriet som startet i $2011$ faldt valget p� at studere metrostationerne og deres effekt p� boligprisen. Min hypotese var at boligerne der l� t�ttest p� de allerede eksisterende metrostationerne alt andet lige ville opleve en mindre positiv effekt end dem der l� nogle meter l�ngere v�k, og jeg satte mig derfor til at besvare dette sp�rgsm�l med mit speciale. 

Tak til min vejleder, S�ren Leth-Petersen for input  og god vejledning. Derudover stor tak til Toke Emil Panduro som har hjulpet mig med datas�ttet og sparring i forbindelse med den additive metode. En stor tak skal der ogs� lyde til IFRO for at l�ne mig datas�ttet, som er brugt i mit speciale. Tak til Bolig�konomisk Videncenters medarbejder for gode r�d og sparring, gennem hele processen. 

En s�rlig tak skal der lyde til menneskerne omkring mig og min familie, som har hjulpet mig gennem min studietid og ikke mindst i specialeprocessen.  
\

\begin{flushright}
Zarah Katharina Saxil Andersen

Maj $2014$
\end{flushright}
