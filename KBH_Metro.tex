\section{Den k�benhavnske metro}

Da metrosystemet i K�benhavn spiller en v�sentlig rolle i den p�g�ldende 
analyse, gives her en n�rmere beskrivelse af, hvorn�r arbejdet med metroens f�rste $22$ stationer 
blev p�begyndt, og hvorn�r de forskellige stationer, der indg�r i analysen, 
havde deres officielle �bningsdag. Specialet besk�ftiger sig ikke med den 
nyligt p�begyndte udvidelse af den K�bnehavnske metro, hvor byggepladserne 
har v�ret etableret siden $2011$.   

Folketinget traf tilbage i $1992$, en beslutning om, at K�benhavn skulle have 
et metrosystem. I $1995$ begyndte udvidelsen af det offentlige 
transportsystem i K�benhavn. Metrobyggeriet blev udf�rt i etaper, hvor den 
f�rste del stod f�rdig i $2002$, mens den sidste del blev f�rdig del i $2007$
. Den k�benhavnske metro har i dag 22 metrostationer, hvoraf 9 af dem er 
underjordiske. I $2018$ vil udvidelsen af metrosystemet st� f�rdigt, og 
metroen vil da have $35$ stationer.   

Metroens allerede eksisterende system, er en stor forbedring af det 
offentlige transportsystem. Metroen er yderst driftssikker og med 13.000 
daglige afgange og en punktlighed p� 98,2 procent, er det et yderst 
velfungerende transportmiddel, som giver en stor v�rdi for de borgere, der 
benytter sig af systemet (Transportministeriet, $2014$).

De f�rste stationer der blev �bnet i 2002 var: N�rreport, Kongens Nytorv, 
Christianshavn, Amagerbro, Lergravsparken, Islands Brygge, DR Byen, Sundby, 
Bella Center, �restad og Vestamager. I 2003 �bnede Forum, Frederiksberg, 
Fasanvej, Lindevang, Flintholm og Vanl�se. De sidste fem stationer: �resund, 
Amager Strand, Fem�ren, Kastrup og Lufthavnen �bnede i 2007.  
 

P� kortet nedenfor er vist en oversigt over metrosystemet i dag. 
\begin{figure}[h]
	\centering
		\includegraphics[scale=0.4]{Zonekort-Metroen}
\end{figure}




