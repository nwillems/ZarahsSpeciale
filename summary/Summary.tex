\section{Executive Summary}
The aim of the thesis is to empirically investigate the price effect on 
apartments in Copenhagen after opening of the metro, specifically the dependance
on distance to the nearest station.

The thesis is motivated by the inconclusive results provided by current 
literature, investigating the same effect on apartment prices in other larger 
cities. 

Agostini and Palmucci($2008$) proves a decreasing relation between 
distance and prices, while Chen et al. ($1997$) proves that the same effect is 
not present for the houses nearest to the stations, while maintaining similar 
properties for the remaining houses.

The dataset used in the analysis is covering all apartments sold more than once 
during the period January $2000$ to April $2013$, in Copenhagen, Frederiksberg 
excluded. The dataset is separated in two periods, the time before and after the 
metro station nearest the apartment were opened. The analysis does not account 
for the announcement- or negative effect caused by the construction sites 
due to the metro expansion.  
 
The analysis includes three groups of variables: \textit{Characteristics}, 
\textit{Spatial} and \textit{Neighborhood}. \textit{Characteristics} describes the apartment. It is variables like size in 
square meters, number of rooms, age of the building, units in the building and 
floor. \textit{Spatial} explains distance to the metro station, park and 
daycare. \textit{Neighborhood} includes average income in the apartments 
and school ranking, based on the average of grade-average for graduates. 


The analysis is done by setting up two econometric models:  A 
hedonic price effect model based on Rosen's theory from $1974$ and a additive model 
inspired by Hastie and Tibshirani( $1990$). 

The hedonic price model fit a function of residential prices, using the natural 
logarithm to a set of co-variates. 

The additive model utilizes cubic smoothing splines to include a non-linear effect
and give more flexibility to the model, to better capture the spatial dependence. 

The additive model includes two concepts ``spline functions'' and ``smoothers''.  
A spline function is used in the construction of a spline curve, where the 
spline curve is used to create a smoothing curve. The spline curve is a curve 
reacted by a piece wise polynomial, where two or more polynomials are combined 
to create a curve. The smoother is a technique to summarize the trend for a 
response variable $Y$ and a function of one or more explanatory variables. The 
practical by a smoothing is that it does not use any fixed form on the 
explanatory variables.  An example of a smoother is the 
``Moving average''-technique. The smoother is simply building upon the idea that 
the dataset is split into pieces and on each piece of the dataset is 
fited an average, simple regression or polynomials, differ from the technique
there is used.

An important issue in the analysis is the trade-off between variance and 
bias, which plays an important role when the dataset is split into smaller 
intervals. A large interval produces a curve with high degree of smoothing, 
and estimates with low variance and high bias, because of high degree of 
smoothing and therefore greater distance between actual curve and smoothing 
curve.  The opposite will happen with small intervals, where the variance is 
high and the bias small.  In the additive model, are used a function called 
cubic smoothing spline, to balance the trade-off between variance and 
bias, is included a parameter that accounts for that. 

The results from the hedonic model shows a decreasing relation between house prices and 
meters to a metro station. Apartment sales between $0$ and $200$ meters to a 
station, have experienced a high increase in the house prices after the 
station was opened. It is in that relation important to notice that some of 
the 
effect, can simply be because this apartments also was the one suffering the 
most from the noise and dust from the construction site, and therefore have
the prices on this apartment, been pushed down even more under the building 
process, and therefore will some of the raise in prices be related to the 
removing of the construction site, and not because the apartment, now is 
closer to a metro station.

The results from the additive model, shows that the negative externalities by 
living close to a metro station matters. The model proves an upwards trend 
from $0$ to $300$ meters from a metro station, while after $300$ meters to the station, 
the effect on residential prices is decreasing if the apartment it placed further away 
from the metro station.   

The thesis is structured as follows: The first chapter gives a short
introduction to the Copenhagen metro system, the second chapter explains the 
theory behind the hedonic price model and the additive model, and also includes
a short model explaining how distance to public transport is affecting the 
house prices. Chapter three gives a description of the dataset, which goes over
in chapter four which gives an explanation of the variables constructed to use 
in the analysis. Chapter five to seven gives a closer description of the three 
different types of variables. Chapter eight describes the method behind the 
models. Chapter nine presents the estimation results. Chapter ten and eleven 
is summing up the thesis in a discussion of a closing conclusion. 
