\section{Executive Summary}
The aim of the thesis is to empirically investigate the price effect on 
apartments in Copenhagen after opening of the metro, specifically the dependance
on distance to the nearest station.

The thesis is motivated by the inconclusive results provided by current 
literature, investigating the same effect on apartment prices in other larger 
cities. 

Agostini and Palmucci($2008$) proves a decreasing relation between 
distance and prices, while Chen et al. ($1997$) proves that the same effect is 
not present for the houses nearest to the stations, while maintaining similar 
properties for the remaining houses.

The dataset used in the analysis is covering all apartments sold more than once 
during the period January $2000$ to April $2013$, in Copenhagen, Frederiksberg 
excluded. The dataset is separated in two periods, the time before and after the 
metro station nearest the apartment were opened. The analysis does not account 
for the announcement- or negative effect caused by the construction sites 
due to the metro expansion since $2011$.  
 
The analysis includes three groups of variables: \textit{Characteristics}, 
\textit{Spatial} and \textit{Neighborhood}. \textit{Characteristics} describes the apartment, it is variables like size in 
square meters, number of rooms, age of the building, units in the building and 
floor. \textit{Spatial} explains distance to the metro station, parks and 
daycare. \textit{Neighborhood} includes average income in the apartments 
and school ranking, based on the average of grade-average for graduates. 


The analysis is done by setting up two econometric models:  A 
hedonic price effect model based on Rosen's theory from $1974$ and an additive model 
inspired by Hastie and Tibshirani( $1990$). 

The hedonic price model fits a function of residential prices, using the natural 
logarithm to a set of co-variates. 

The additive model utilizes cubic smoothing splines to include a non-linear effect
and give more flexibility to the model, to better capture the spatial dependence. 

The additive model includes two concepts ``spline functions'' and ``smoothers''.  
A spline function is used in the construction of a spline curve, where the 
spline curve is used to create a smoothing curve. The spline curve is a curve 
created by a piecewise polynomial, where two or more polynomials are combined 
to create a curve. The smoother is a technique to summarize the trend for a 
response variable $Y$ and a function of one or more explanatory variables. 
Smothers does not rely on a fixed form for the explanatory variables, thereby 
allowing a non-linear relationship. An example smoother is the 
``Moving average''-technique. Smoothers are building upon the idea, that a 
dataset is split into smaller intervals, and each group is then fitted based on
either average, simple regression or polynomials.

An important issue in the analysis is the trade-off between variance and 
bias, which plays an important role when the dataset is split into smaller 
intervals. A large interval produces a curve with high degree of smoothing, 
and estimates with low variance and high bias, because of high degree of 
smoothing and therefore greater distance between actual curve and smoothing 
curve.  The opposite will happen with small intervals, where the variance is 
high and the bias small.  The additive model, uses a cubic smoothing spline 
function. A parameter to balance between bias and variance is included in the 
smoothing function

The results from the Hedonic model shows a decreasing relation between house prices and 
meters to a metro station. Apartment sales between $0$ and $200$ meters to a 
station, have experienced a high increase in the house prices after the 
station was opened. It is important to note, that some parts of this 
relationship can simply be due to lower prices during construction, which the 
apartments further away has not been suffering from, and thereby not from being
closer to the metro station.

The results from the additive model, shows that the negative externalities by 
living close to a metro station matters. The model proves an upwards trend 
from $0$ to $300$ meters from a metro station, while after $300$ meters to the station, 
the effect on residential price is decreasing if the apartment it placed further away 
from the metro station.   

The thesis is structured as follows: The first section gives a short
introduction to the Copenhagen metro system. The second section explains the 
theory behind the Hedonic price model and the additive model, and also includes
a short model explaining the relation between distance to public transport and
house prices. Section three describes the dataset. Section four explains the 
computed variables used in the analysis. Section $5-7$ goes into more details regarding the three 
types of variables. Section eight describes the practical method for the models. 
Section nine presents the estimation results and section ten and eleven 
sums up the thesis in a discussion and a conclusion. 
