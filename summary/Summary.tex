\section{Executive Summary}
The aim of the thesis is to empirically investigate the price effect on 
apartments in Copenhagen after opening of the first $22$ metro stations, and 
specifically, how the effect is depending on the distance to the nearest 
metro station. The analysis is done by setting up two econometric models:  A 
hedonic price effect model based on Rosen's theory from $1974$ which fits a 
function of house prices, in the analysis is used the natural logarithm to a 
set of co-variates,  and an additive model using cubic smoothing splines to 
include a non-linear effect and give more flexibility to the model, to better 
capture the spatial dependence. The additional model is inspired by Hastie and 
Tibshirani( $1990$). 

The two models are estimated using a dataset covering all sales for 
apartments sold more than one time doing the period $2000$ to April $2013$. The 
dataset is separated in to periods, the time before and after the metro 
station closes to the apartment was opened. Since $2005$ it has been decided 
that the metro system should be expanded with an extra line, and from $2011$ 
the contractions site was established. The dataset could consist apartment 
sales there is positive affect by a announcement effect or negative affected 
be the noise from a construction site, but the analysis does not take account 
for the new expanding of the metro system, it only looks at  the finish $22$ 
stations, and how there have affected the apartment sales in Copenhagen, 
exclusive Frederiksberg. 

The analysis includes three groups of variables: characteristics variable, 
spatial variable and neighborhood variables. The characteristics variable 
include variable describing the apartment. It is variable like size in square 
meters , number of room, age of the building, units in the building and 
floor. The spatial variables explain distance to the metro station, park and 
daycare. The neighborhood variables include average income in the apartments 
and school ranking, based on the average of grade-average for graduates. 

The additive model uses a technique to fit non-linear relationships to 
multivariate data. The additive model includes to concepts "`spline functions"
' and "`smoothers"'.  A spline function is used in the construction of a 
spline curve, where the spline curve is used to create at smoothing curve.  
The spline curve is a curve reacted by piece wise polynomial, where two or 
more polynomial a combined to create a curve. The smoother is a technic to 
summarize the trend for a response variable Y ad a function of one or more 
explanatory variables. The practical by a smoothing is that is does not use 
any fix form on the explanatory variables.  An example of a smoother is the 
"`Moving average- technique"'. The smoother is simple building upon the idea 
that the dataset is split into pieces and on each pieces of the dataset is 
fit an average,  simple regression or polynomials, differ from the technic 
there is used.  

And important issue in the analysis is the trade-off between variance and 
bias, which play a important role when the dataset is split into smaller 
intervals. A large interval produce a curve with high degree of smoothing, an 
estimates with low variance and high bias, because of high degree of 
smoothing and therefore greater distance between actual curve and smoothing 
curve.  The opposite will happen with a small interval, where the variance is 
high and the bias small.  In the additive model, is used a function called 
cubic smoothing spline, and to balance the trade-off between variance and 
bias, is included a parameter there can account for that. 

The motivation for investigate what effect depending on meters to the 
stations there is related to the metro station is grounded in different 
relation explaining in the literature. Doing the positive and negative 
externalities related to living close to a metro stations, studies of 
relation between house prices and living close to public transport have 
revealed different relations. 

So fare Agostini and Palmucci($2008$) prove a linear relation while Chen et al (
 $1997$) proof that it is possible to observe a lower price for house just next 
to the station, and after would a raise in the price, and as the houses move 
further away from the station, there will be observed a downward trend. 

The hedonic models estimates show a linear relation between house prices and 
meter to a metro station. Apartments sales between $0$ and $200$ meters to a 
station, have experience a high increase in the house prices after the 
station was open. It is in that relation important to notice that some of the 
effect, can simply be because this apartments also was the one suffering the 
most from the noise and dust from the construction site, and there for have 
the prices on this apartment, been pushed done even more under the building 
process, and there for will some of the raise in prices be related to the 
removing of the construction site, and not because the apartment, now is 
closer to a metro station.   

The additive model, which gives additional flexibility permits more accurate 
estimation of the underlying spatial structure of the date. This should leed 
to more reliable results for the impact of having a apartments close to a 
metro station.  
The estimates for the additive model, shows that the negativ eksternalites by 
living close to a metro station matters. The model proof a upgoing trend from $0$
to 300 meters from a metro station, while after $300$ meters to the station, 
the effect on house prices is falling af the apartment it place further away 
from the metro station.   

The thesis is structured as follow: The first chapter gives a short introducing 
to the Copenhagen metro system, the second chapter explain the theory behind the
hedonic price model and the additive model, and also include a short model 
explaining how distance to public transport is affecting the house prices. 
Chapter three gives a description of the data, which goes over in chapter four
which gives a explanation of the variables constructed to used in the analysis. 
Chapter five to seven gives a closer description of the three different types of 
variables. Chapter eight describe the method behind the models. Chapter nine 
present the estimation results. Chapter ten and eleven is summing up the thesis 
in a discussion of a closing conclusion. 


   
